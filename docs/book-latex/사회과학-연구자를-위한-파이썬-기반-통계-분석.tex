% Options for packages loaded elsewhere
% Options for packages loaded elsewhere
\PassOptionsToPackage{unicode}{hyperref}
\PassOptionsToPackage{hyphens}{url}
%
\documentclass[
  letterpaper,
]{book}
\usepackage{xcolor}
\usepackage{amsmath,amssymb}
\setcounter{secnumdepth}{5}
\usepackage{iftex}
\ifPDFTeX
  \usepackage[T1]{fontenc}
  \usepackage[utf8]{inputenc}
  \usepackage{textcomp} % provide euro and other symbols
\else % if luatex or xetex
  \usepackage{unicode-math} % this also loads fontspec
  \defaultfontfeatures{Scale=MatchLowercase}
  \defaultfontfeatures[\rmfamily]{Ligatures=TeX,Scale=1}
\fi
\usepackage{lmodern}
\ifPDFTeX\else
  % xetex/luatex font selection
  \setmainfont[]{Spoqa-Han-Sans-Neo}
\fi
% Use upquote if available, for straight quotes in verbatim environments
\IfFileExists{upquote.sty}{\usepackage{upquote}}{}
\IfFileExists{microtype.sty}{% use microtype if available
  \usepackage[]{microtype}
  \UseMicrotypeSet[protrusion]{basicmath} % disable protrusion for tt fonts
}{}
\makeatletter
\@ifundefined{KOMAClassName}{% if non-KOMA class
  \IfFileExists{parskip.sty}{%
    \usepackage{parskip}
  }{% else
    \setlength{\parindent}{0pt}
    \setlength{\parskip}{6pt plus 2pt minus 1pt}}
}{% if KOMA class
  \KOMAoptions{parskip=half}}
\makeatother
% Make \paragraph and \subparagraph free-standing
\makeatletter
\ifx\paragraph\undefined\else
  \let\oldparagraph\paragraph
  \renewcommand{\paragraph}{
    \@ifstar
      \xxxParagraphStar
      \xxxParagraphNoStar
  }
  \newcommand{\xxxParagraphStar}[1]{\oldparagraph*{#1}\mbox{}}
  \newcommand{\xxxParagraphNoStar}[1]{\oldparagraph{#1}\mbox{}}
\fi
\ifx\subparagraph\undefined\else
  \let\oldsubparagraph\subparagraph
  \renewcommand{\subparagraph}{
    \@ifstar
      \xxxSubParagraphStar
      \xxxSubParagraphNoStar
  }
  \newcommand{\xxxSubParagraphStar}[1]{\oldsubparagraph*{#1}\mbox{}}
  \newcommand{\xxxSubParagraphNoStar}[1]{\oldsubparagraph{#1}\mbox{}}
\fi
\makeatother


\usepackage{longtable,booktabs,array}
\usepackage{calc} % for calculating minipage widths
% Correct order of tables after \paragraph or \subparagraph
\usepackage{etoolbox}
\makeatletter
\patchcmd\longtable{\par}{\if@noskipsec\mbox{}\fi\par}{}{}
\makeatother
% Allow footnotes in longtable head/foot
\IfFileExists{footnotehyper.sty}{\usepackage{footnotehyper}}{\usepackage{footnote}}
\makesavenoteenv{longtable}
\usepackage{graphicx}
\makeatletter
\newsavebox\pandoc@box
\newcommand*\pandocbounded[1]{% scales image to fit in text height/width
  \sbox\pandoc@box{#1}%
  \Gscale@div\@tempa{\textheight}{\dimexpr\ht\pandoc@box+\dp\pandoc@box\relax}%
  \Gscale@div\@tempb{\linewidth}{\wd\pandoc@box}%
  \ifdim\@tempb\p@<\@tempa\p@\let\@tempa\@tempb\fi% select the smaller of both
  \ifdim\@tempa\p@<\p@\scalebox{\@tempa}{\usebox\pandoc@box}%
  \else\usebox{\pandoc@box}%
  \fi%
}
% Set default figure placement to htbp
\def\fps@figure{htbp}
\makeatother





\setlength{\emergencystretch}{3em} % prevent overfull lines

\providecommand{\tightlist}{%
  \setlength{\itemsep}{0pt}\setlength{\parskip}{0pt}}



 


\makeatletter
\@ifpackageloaded{bookmark}{}{\usepackage{bookmark}}
\makeatother
\makeatletter
\@ifpackageloaded{caption}{}{\usepackage{caption}}
\AtBeginDocument{%
\ifdefined\contentsname
  \renewcommand*\contentsname{Table of contents}
\else
  \newcommand\contentsname{Table of contents}
\fi
\ifdefined\listfigurename
  \renewcommand*\listfigurename{List of Figures}
\else
  \newcommand\listfigurename{List of Figures}
\fi
\ifdefined\listtablename
  \renewcommand*\listtablename{List of Tables}
\else
  \newcommand\listtablename{List of Tables}
\fi
\ifdefined\figurename
  \renewcommand*\figurename{Figure}
\else
  \newcommand\figurename{Figure}
\fi
\ifdefined\tablename
  \renewcommand*\tablename{Table}
\else
  \newcommand\tablename{Table}
\fi
}
\@ifpackageloaded{float}{}{\usepackage{float}}
\floatstyle{ruled}
\@ifundefined{c@chapter}{\newfloat{codelisting}{h}{lop}}{\newfloat{codelisting}{h}{lop}[chapter]}
\floatname{codelisting}{Listing}
\newcommand*\listoflistings{\listof{codelisting}{List of Listings}}
\makeatother
\makeatletter
\makeatother
\makeatletter
\@ifpackageloaded{caption}{}{\usepackage{caption}}
\@ifpackageloaded{subcaption}{}{\usepackage{subcaption}}
\makeatother
\usepackage{bookmark}
\IfFileExists{xurl.sty}{\usepackage{xurl}}{} % add URL line breaks if available
\urlstyle{same}
\hypersetup{
  pdftitle={사회과학 연구자를 위한 파이썬 기반 통계 분석},
  pdfauthor={당신의 이름},
  hidelinks,
  pdfcreator={LaTeX via pandoc}}


\title{사회과학 연구자를 위한 파이썬 기반 통계 분석}
\author{당신의 이름}
\date{2025-04-21}
\begin{document}
\frontmatter
\maketitle

\renewcommand*\contentsname{Table of contents}
{
\setcounter{tocdepth}{2}
\tableofcontents
}

\mainmatter
\bookmarksetup{startatroot}

\chapter*{맞이하는 글}\label{uxb9deuxc774uxd558uxb294-uxae00}
\addcontentsline{toc}{chapter}{맞이하는 글}

\markboth{맞이하는 글}{맞이하는 글}

사회과학 연구는 복잡다단한 사회 현상의 심층적 이해와 과학적 설명을
목표로 끊임없이 정진해 왔습니다. 급증하는 데이터의 양과 비약적으로
발전하는 정보 기술은 사회과학 연구의 새로운 패러다임을 제시하며,
그중에서도 파이썬은 고성능 데이터 분석 도구와 광범위한 라이브러리
생태계를 기반으로 사회과학 연구자에게 필수적인 분석 역량으로 부상하고
있습니다.

본 저서는 \textbf{인문 사회 분야 연구자}들이 기존의 상용 통계 패키지의
제한적인 환경에서 벗어나, \textbf{파이썬을 활용하여 보다 유연하고
효율적인 분석 체계를 구축}하고, \textbf{표준화된 코드를 통해 분석 전
과정을 체계적으로 관리하며, 시각화된 결과물을 논문 작성의 근거 자료로
활용}하는 데 실질적인 도움을 제공하고자 기획되었습니다.

기존의 통계학 교재들이 이론적 논의에 치중하거나, 특정 통계 패키지의
조작법만을 단편적으로 기술한 데 반하여, 본 저서는 가상 데이터를 기반으로
각 통계 분석 기법의 \textbf{이론적 토대}와 \textbf{파이썬 라이브러리를
이용한 실제 분석 절차}를 통합적으로 제시합니다. 독자들은 다양한 형태의
가상 데이터를 직접 조작하고, 파이썬 코드를 실행하여 도출된 결과를
해석하는 실습 과정을 통해 통계 분석 능력을 체계적으로 함양할 수
있습니다. 더불어, 각 분석 방법론이 사회과학 연구의 다양한 영역에서
어떻게 적용될 수 있는지 구체적인 사례를 제시함으로써 학습의 응용
가능성을 제고하였습니다.

본 저서를 통해 독자들이 파이썬 기반의 통계 분석에 대한 확신을 얻고,
\textbf{데이터 기반의 객관적이고 과학적인 사회과학 연구를 수행하며
학문적 성과의 완성도를 제고하는 데 필요한 핵심 역량을 확보}하게 되기를
기대하는 바입니다.

선행학습 : 파이썬 문법 기초 및 자료형

\bookmarksetup{startatroot}

\chapter*{목차}\label{uxbaa9uxcc28}
\addcontentsline{toc}{chapter}{목차}

\markboth{목차}{목차}

\textbf{제1부: 파이썬 기반 사회과학 데이터 분석 준비}

\begin{itemize}
\tightlist
\item
  제1장: 사회과학 연구와 데이터의 이해

  \begin{itemize}
  \tightlist
  \item
    과학적 연구 방법론의 기초
  \item
    사회과학 데이터의 유형과 특성 (설문, 실험, 관찰 데이터 등)
  \item
    가상 데이터의 효용성과 장점
  \item
    파이썬 통계 분석 환경 구축 (Anaconda, Jupyter Notebook 소개)
  \item
    주요 파이썬 라이브러리 개요 (Pandas, NumPy, Matplotlib, Seaborn)
  \end{itemize}
\item
  제2장: 파이썬을 이용한 데이터 핸들링 및 탐색

  \begin{itemize}
  \tightlist
  \item
    Pandas DataFrame 기초: 데이터 로딩, 저장, 구조 이해
  \item
    데이터 전처리: 결측값 처리, 이상치 탐지 및 조정, 변수 변환
  \item
    기술 통계 분석: 평균, 표준편차, 빈도 분석 등
  \item
    데이터 시각화: Matplotlib, Seaborn 활용 (히스토그램, 산점도, 막대
    그래프 등)
  \item
    \textbf{가상 데이터 예시:} 설문조사 응답 데이터 (개인 속성, 태도,
    행위 관련 변수 포함)
  \end{itemize}
\end{itemize}

\textbf{제2부: 집단 간 차이 검증 및 관계 분석}

\begin{itemize}
\tightlist
\item
  제3장: 집단 간 평균 차이 검증

  \begin{itemize}
  \tightlist
  \item
    독립표본 t-검정: 두 독립 집단의 평균 비교
  \item
    대응표본 t-검정: 동일 집단의 사전-사후 평균 비교
  \item
    일원분산분석 (ANOVA): 세 개 이상 집단의 평균 비교
  \item
    사후 검정 (Post-hoc test)
  \item
    \textbf{가상 데이터 예시:}

    \begin{itemize}
    \tightlist
    \item
      상이한 교육 방식의 효과 비교 (학업 성취도 데이터)
    \item
      특정 사건 전후의 태도 변화 측정 데이터
    \item
      복수 지역 간 소득 수준 비교 데이터
    \end{itemize}
  \end{itemize}
\item
  제4장: 범주형 데이터 분석 및 연관성 분석

  \begin{itemize}
  \tightlist
  \item
    교차 분석 (Cross-tabulation): 범주형 변수 간 빈도 분석
  \item
    카이제곱 검정: 범주형 변수 간 독립성 검증
  \item
    피셔의 정확 검정
  \item
    \textbf{가상 데이터 예시:}

    \begin{itemize}
    \tightlist
    \item
      성별과 특정 정치적 성향 간의 관계 분석 (설문 데이터)
    \item
      광고 노출 여부와 제품 구매 여부 간의 관계 분석
    \end{itemize}
  \end{itemize}
\item
  제5장: 변수 간 상관관계 분석

  \begin{itemize}
  \tightlist
  \item
    피어슨 상관분석: 연속형 변수 간 선형적 관계 측정
  \item
    스피어만 상관분석: 순위형 변수 간 관계 측정
  \item
    편상관분석: 통제 변수 고려 하 두 변수 간 순수 상관관계 분석
  \item
    \textbf{가상 데이터 예시:}

    \begin{itemize}
    \tightlist
    \item
      소득 수준과 주관적 행복감 간의 관계 분석 (설문 데이터)
    \item
      광고 지출액과 매출액 간의 관계 분석 (시계열 또는 횡단면 데이터)
    \end{itemize}
  \end{itemize}
\end{itemize}

\textbf{제3부: 예측 및 설명 모형화}

\begin{itemize}
\tightlist
\item
  제6장: 선형 회귀분석

  \begin{itemize}
  \tightlist
  \item
    단순 선형 회귀분석: 단일 독립변수의 종속변수에 대한 영향 분석
  \item
    다중 선형 회귀분석: 복수 독립변수의 종속변수에 대한 영향 분석
  \item
    회귀 모형 평가 (결정 계수, F-통계량)
  \item
    회귀 진단 (잔차 분석, 다중공선성)
  \item
    \textbf{가상 데이터 예시:}

    \begin{itemize}
    \tightlist
    \item
      광고비, 소득 수준의 제품 구매량에 대한 영향 분석
    \item
      학습 시간, 학습 방법의 시험 성적에 대한 영향 분석
    \end{itemize}
  \end{itemize}
\item
  제7장: 로지스틱 회귀분석

  \begin{itemize}
  \tightlist
  \item
    이항 로지스틱 회귀분석: 이분형 종속변수 예측
  \item
    다항 로지스틱 회귀분석: 다범주 종속변수 예측
  \item
    오즈비 (Odds Ratio) 해석
  \item
    모형 평가 (정확도, ROC 곡선)
  \item
    \textbf{가상 데이터 예시:}

    \begin{itemize}
    \tightlist
    \item
      개인 속성이 특정 정치 후보 지지 여부에 미치는 영향 분석
    \item
      마케팅 캠페인 요소가 고객의 구매 전환 여부에 미치는 영향 분석
    \end{itemize}
  \end{itemize}
\item
  제8장: 의사결정나무 모형

  \begin{itemize}
  \tightlist
  \item
    분류 나무 (Classification Tree): 범주형 종속변수 예측
  \item
    회귀 나무 (Regression Tree): 연속형 종속변수 예측
  \item
    가지치기 (Pruning)
  \item
    변수 중요도 분석
  \item
    \textbf{가상 데이터 예시:}

    \begin{itemize}
    \tightlist
    \item
      고객 특성에 따른 상품 추천 모형 개발
    \item
      학생 성적 예측 모형 개발
    \end{itemize}
  \end{itemize}
\item
  제9장: 군집 분석

  \begin{itemize}
  \tightlist
  \item
    비계층적 군집 분석 (K-평균 군집)
  \item
    계층적 군집 분석 (병합적, 분할적 방법)
  \item
    군집 평가
  \item
    \textbf{가상 데이터 예시:}

    \begin{itemize}
    \tightlist
    \item
      소비자 행태 패턴 기반 시장 세분화
    \item
      지역별 사회 경제적 특성 기반 군집화
    \end{itemize}
  \end{itemize}
\item
  제10장: 요인 분석

  \begin{itemize}
  \tightlist
  \item
    탐색적 요인 분석: 잠재 변수 (요인) 추출
  \item
    확인적 요인 분석 (간략히 소개)
  \item
    요인 회전
  \item
    요인 점수
  \item
    \textbf{가상 데이터 예시:}

    \begin{itemize}
    \tightlist
    \item
      소비자 브랜드 태도 측정 설문 데이터의 주요 요인 추출
    \item
      직무 만족도 설문 데이터의 하위 요인 구조 파악
    \end{itemize}
  \end{itemize}
\end{itemize}

\textbf{제4부: 심화 분석 기법 (선택적)}

\begin{itemize}
\tightlist
\item
  제11장: 구조방정식 모형 (SEM) 개요

  \begin{itemize}
  \tightlist
  \item
    경로 분석
  \item
    잠재 변수 모형
  \item
    모형 식별 및 적합도 평가 (개략적 소개)
  \item
    \textbf{가상 데이터 예시:} 사회적 지지와 우울증 간 관계에서 자존감의
    매개 효과 분석 (가상 경로 모형 제시)
  \end{itemize}
\item
  제12장: 다차원 척도법 (MDS)

  \begin{itemize}
  \tightlist
  \item
    객체 간 유사성/비유사성 데이터 분석
  \item
    저차원 공간에서의 객체 시각화
  \item
    \textbf{가상 데이터 예시:} 다양한 브랜드 간 이미지 유사성 평가
    데이터 분석
  \end{itemize}
\item
  제13장: 상응 분석

  \begin{itemize}
  \tightlist
  \item
    범주형 변수 간 관계 시각화
  \item
    행 및 열 프로필 분석
  \item
    \textbf{가상 데이터 예시:} 제품 범주와 소비자 특성 간 관계 분석
  \end{itemize}
\item
  제14장: 컨조인트 분석 (Conjoint Analysis)

  \begin{itemize}
  \tightlist
  \item
    속성 수준 조합에 대한 선호도 분석
  \item
    각 속성 수준의 중요도 파악
  \item
    \textbf{가상 데이터 예시:} 신제품의 다양한 속성 조합에 대한 소비자
    선호도 분석
  \end{itemize}
\item
  제15장: 시계열 분석 기초 (선택적)

  \begin{itemize}
  \tightlist
  \item
    시계열 데이터의 이해
  \item
    추세 및 계절성 분석 (개략적 소개)
  \item
    ARIMA 모형 기초 (개략적 소개)
  \item
    \textbf{가상 데이터 예시:} 월별 판매량 데이터 분석
  \end{itemize}
\end{itemize}

\textbf{부록:}

\begin{itemize}
\tightlist
\item
  주요 파이썬 라이브러리 함수 요약
\item
  통계 용어 해설
\item
  연습 문제 및 해답
\end{itemize}

\part{제1부 - 파이썬을 활용한 사회과학 데이터 분석 준비}

\chapter{}\label{section}

\chapter{}\label{section-1}

\part{제2부 - 집단 간 차이 분석 및 관계 분석}

\chapter{}\label{section-2}

\chapter{}\label{section-3}

\chapter{}\label{section-4}

\part{제3부 - 예측 및 설명 모델링}

\chapter{}\label{section-5}

\chapter{}\label{section-6}

\chapter{}\label{section-7}

\chapter{}\label{section-8}

\chapter{}\label{section-9}

\part{제4부 - 심화 분석 기법 (선택적)}

\chapter{}\label{section-10}

\chapter{}\label{section-11}

\chapter{}\label{section-12}

\chapter{}\label{section-13}

\chapter{}\label{section-14}

\cleardoublepage
\phantomsection
\addcontentsline{toc}{part}{Appendices}
\appendix

\chapter{}\label{section-15}

\chapter{}\label{section-16}

\chapter{}\label{section-17}


\backmatter


\end{document}
